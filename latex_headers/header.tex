\newcommand{\thesistitle}{PDF to the people: an atomic pair distribution function analysis primer}
\newcommand{\thesisauthor}{Simon J. L. Billinge, Kirsten M. {\O}. Jensen, Soham Banerjee, Emil S. Bozin, Benjamin A. Frandsen, Maxwell W. Terban, Robert J. Koch }
\newcommand{\thesisyear}{2022}
\newcommand{\chFlag}{hold}
\newcommand{\button}[1]{{\tt #1}}

\newcommand{\plotdata}{{\sc plotdata}\xspace}
\newcommand{\getx}{{\sc PDFgetX3}\xspace}
\newcommand{\gui}{{\sc PDFgui}\xspace}
\newcommand{\itc}{{\sc PDFitc}\xspace}
\newcommand{\guitwo}{{\sc PDFgui2}\xspace}
\newcommand{\xsuite}{{\sc xPDFsuite}\xspace}
\newcommand{\esuite}{{\sc ePDFsuite}\xspace}
\newcommand{\cmi}{{\sc Diffpy-CMI}\xspace}
\newcommand{\morph}{{\sc PDFmorph}\xspace}
\newcommand{\srmise}{{\sc SrMise}\xspace}
\newcommand{\srfit}{{\sc SrFit}\xspace}
\newcommand{\srxplanar}{{\sc SrXplanar}\xspace}
\newcommand{\jupyter}{{\sc Jupyter}\xspace}
\newcommand{\pycharm}{{\sc PyCharm}\xspace}
\newcommand{\xpdacq}{{\sc xpdAcq}\xspace}
\newcommand{\xpdan}{{\sc xpdAn}\xspace}
\newcommand{\xpdtools}{{\sc xpdtools}\xspace}
\newcommand{\shed}{{\sc SHED}\xspace}
\newcommand{\pyfai}{{\sc PyFai}\xspace}
\newcommand{\mpdf}{{\sc diffpy.mpdf}\xspace}
\newcommand{\strumining}{{\sc structureMining}\xspace}
\newcommand{\clustermining}{{\sc clusterMining}\xspace}
\newcommand{\nmfmapping}{{\sc nmfMapping}\xspace}
\newcommand{\motexml}{{\sc MotEx-ML}\xspace}
\newcommand{\gotcha}[1]{\begin{quote} {\bf Gotcha: }#1\end{quote}}
\newcommand{\download}{Follow the instructions in Section~\ref{ssec:chnm-ww-download}.}
\newcommand{\plotgr}[1]{Plot the #1 using your preferred plotting program. See Section~\ref{sec:chnm-e} for help with {\tt .gr} file formats. See Sections~\ref{ssec:chnm-ww-plot} and~\ref{ssec:chgtp-ww-plot} for help with doing this using
Diffpy tools.}
\newcommand{\plotchi}[1]{Plot the #1 files using your preferred plotting program. See Section~\ref{sec:chgtp-e} for help with {\tt .chi}, {\tt .iq}, etc., file formats. See Section~\ref{ssec:chgtp-ww-plot} for help with doing this using
Diffpy tools}
\newcommand{\cmipreamble}{This section describes the \cmi solution.  It may be
skipped on a first reading, but it will be helpful to you when, in the future, you want to learn how to use \cmi.  To get started with \cmi, follow the instructions in Section~\ref{sec:chnm-cmisol}. The \cmi code for the solution will be in a python file in the diffpy-cmi folder under solutions for this chapter in your downloaded data.  Read through the comments there.  They are expanded on below in greater detail where needed.}
\newcommand{\itcref}{Using \strumining in \itc is a great way to get \cif files to make a start on your refinement campaign.  Please see~\ref{ssec:chgen-ww-pdfitc} for more details.
}

% Much-used math
\newcommand{\qmax}{\ensuremath{Q_{\mathrm{max}}}\xspace}
\newcommand{\qmaxinst}{\ensuremath{Q_{\mathrm{max-inst}}}\xspace}
\newcommand{\qmin}{\ensuremath{Q_{\mathrm{min}}}\xspace}
\newcommand{\qbroad}{\ensuremath{Q_{\mathrm{broad}}}\xspace}
\newcommand{\qdamp}{\ensuremath{Q_{\mathrm{damp}}}\xspace}
\newcommand{\deltatwo}{\ensuremath{\delta_2}\xspace}
\newcommand{\deltaone}{\ensuremath{\delta_1}\xspace}
\newcommand{\spdiameter}{\ensuremath{sp_{\mathrm{diameter}}}\xspace}
\newcommand{\uiso}{\ensuremath{U_{\mathrm{iso}}}\xspace}
\newcommand{\uaniso}{\ensuremath{U_{\mathrm{aniso}}}\xspace}
\newcommand{\sratio}{\ensuremath{sratio}\xspace}
\newcommand{\rpoly}{\ensuremath{rpoly}\xspace}
\newcommand{\rw}{\ensuremath{R_w}\xspace}
\newcommand{\rapdf}{\textit{RA-PDF}\space}
\newcommand{\cif}{{\tt CIF}\xspace}
\newcommand{\chifile}{{\tt .chi}\xspace}


\newcommand{\rmin}{\ensuremath{r_{\mathrm{min}}}\xspace}
\newcommand{\rmax}{\ensuremath{r_{\mathrm{max}}}\xspace}
\newcommand{\rlim}{\ensuremath{r_{\mathrm{lim}}}\xspace}
\newcommand{\mmax}{\ensuremath{m_{\mathrm{max}}}\xspace}
\newcommand{\delr}{\ensuremath{dr}\xspace}
\newcommand{\delrn}{\ensuremath{dr_N}\xspace}
\newcommand{\delq}{\ensuremath{dQ}\xspace}
\newcommand{\qualp}{\ensuremath{\mathcal{Q}_p}\xspace}
\newcommand{\iq}{\ensuremath{I(Q)}\xspace}
\newcommand{\sq}{\ensuremath{S(Q)}\xspace}
\newcommand{\fq}{\ensuremath{F(Q)}\xspace}
\newcommand{\gr}{\ensuremath{G(r)}\xspace}
\newcommand{\rr}{\ensuremath{R(r)}\xspace}
\newcommand{\q}{\ensuremath{Q}\xspace}
\newcommand{\ir}{\ensuremath{r}\xspace}

%units
\newcommand{\iaa}{\AA\ensuremath{^{-1}}\xspace}
\newcommand{\iaasquared}{\AA\ensuremath{^{-2}}\xspace}
\newcommand{\aasquared}{\AA\ensuremath{^{2}}\xspace}



\newcommand{\sjb}[1]{\textcolor{blue}{[sjb:#1]}}
\newcommand{\koj}[1]{\textcolor{brown}{[koj:#1]}}
%\newcommand{\sjb}[1]{}
\newcommand{\sob}[1]{\textcolor{red}{[sob:#1]}}
%\newcommand{\soh}[1]{}
\newcommand{\baf}[1]{\textcolor{green}{[baf:#1]}}
%\newcommand{\baf}[1]{}
\newcommand{\rjk}[1]{\textcolor{purple}{[rjk:#1]}}
\newcommand{\sjba}[1]{}
\newcommand{\mt}[1]{\textcolor{orange}{[mt:#1]}}
\newcommand{\mta}[1]{}
\newcommand{\solutions}[1]{#1}
\newcommand{\bs}[1]{\boldsymbol{#1}}
\newcommand{\dd}{\mathrm{d}}
\newcommand{\ii}{\mathrm{i}}
\newcommand{\eg}{e.g.~}
\newcommand{\eq}[1]{Eq.~\ref{eq:#1}}
\newcommand{\eqs}[1]{Eqs.~\ref{eq:#1}}
\newcommand{\fig}[1]{Fig.~\ref{fig:#1}}
\newcommand{\figs}[1]{Figs.~\ref{fig:#1}}
\newcommand{\sect}[1]{Section~\ref{sec:#1}}
\newcommand{\apen}[1]{Appendix~\ref{sec:#1}}
\newcommand{\tabl}[1]{Table~\ref{table:#1}}
\newcommand{\bra}[1]{\left<#1\right|}
\newcommand{\ket}[1]{\left|#1\right>}
\newcommand{\abs}[1]{\left|#1\right|}
\newcommand{\etal}{{\it et al.~}}
\newcommand{\adhoc}{{\it ad hoc}\xspace}
\newcommand{\nb}[1]{{\bf NOTE[#1]}}
\newcommand{\nba}[1]{}
\newcommand{\mAA}{{\text~\AA}\xspace}

\renewcommand{\vec}[1]{\mathbf{#1}}

%%%
%%% Packages
%%%
\usepackage[dvips]{epsfig}
\usepackage{graphicx}% Include figure files
\usepackage{amsmath}
\usepackage{amssymb}
\usepackage{commath} %http://ctan.mackichan.com/macros/latex/contrib/commath/commath.pdf
\usepackage{xspace}
\usepackage{named}
\usepackage{fancyhdr}
\usepackage{afterpage}
\usepackage{xcolor}
\usepackage{dcolumn}
\usepackage{comment}
\usepackage{listings}
\usepackage{makeidx}
\graphicspath{ {./figures/} }
\usepackage{longtable}



% Colored symbols

\definecolor{dgreen}{HTML}{008000}
\newcommand{\done}{\textcolor{dgreen}{{\large{$\checkmark$}}}}
\newcommand{\doing}{\textcolor{blue}{{\large{$\blacklozenge$}}}}
\newcommand{\checkme}{\textcolor{orange}{{\large{$\blacktriangleleft$}}}}
\newcommand{\fixme}{\textcolor{red}{{\large{$\blacktriangleright$}}}}
%
% We use the hyperref package and customize it for optimal PDF
%
%\usepackage[dvipdfm,pdftitle={\thesistitle},pdfauthor={\thesisauthor},pdfpagemode={UseOutlines},letterpaper,bookmarks,bookmarksopen=true,pdfstartview={FitH},bookmarksnumbered=true,]{hyperref}
\usepackage[pdftitle={\thesistitle},pdfauthor={\thesisauthor},pdfpagemode={UseOutlines},letterpaper,bookmarks,bookmarksopen=true,pdfstartview={FitH},bookmarksnumbered=true,]{hyperref}
%%%
%%% Margins
%%%
\paperwidth=8.5in
\paperheight=11in

% 1in + hoffset + oddsidemargin + textwidth + marginparsep + marginparwidth
% For PhD at Columbia we have single side theses and 1.5in left margin
% The settings below leave 1.5 inch margin at the left and 1 inch at the right
% for US Letter paper
\setlength{\hoffset}{0.0in}
\setlength{\oddsidemargin}{.5in}
\setlength{\textwidth}{6in}
\setlength{\evensidemargin}{0mm}

% 1in + voffset + topmargin + headheight + headsep + textheight + footskip
% For PhD thesis we also need an extra inch at the bottom
% 1inch = 72 pt
\setlength{\voffset}{0.0in}
\setlength{\topmargin}{.0in}
\setlength{\headheight}{14pt}
\setlength{\headsep}{22pt}
\setlength{\textheight}{8.5in}
\setlength{\footskip}{0pt}

%%%
%%% Spacing
%%%
\newcommand{\singlespace}{\renewcommand{\baselinestretch}{1.15} \small \normalsize}
\newcommand{\oneandhalfspace}{\renewcommand{\baselinestretch}{1.3} \small \normalsize}
\newcommand{\doublespace}{\renewcommand{\baselinestretch}{1.7} \small \normalsize}
\newcommand{\normalspace}{\doublespace}
\footnotesep=1\baselineskip

%%%
%%% Counters depth
%%%
\setcounter{secnumdepth}{3}
\setcounter{tocdepth}{3}

%%%
%%% Title page.
%%%
\newcommand{\thesistitlepage}{
    \normalspace
    \thispagestyle{empty}
    \begin{center}
        \textbf{\LARGE \thesistitle} \\[1cm]
        \textbf{\LARGE \thesisauthor} \\[8cm]
        \ \\
        \ \\
        \ \\
        \ \\[3cm]
        \textbf{\Large COLUMBIA UNIVERSITY} \\[4mm]
        \thesisyear
    \end{center}
    \clearpage
}

%%%
%%% Copyright page.
%%%
\newcommand{\thesiscopyrightpage}{
    \thispagestyle{empty}
    \strut \vfill
    \begin{center}
      \copyright \thesisyear \\
      \thesisauthor \\
      All Rights Reserved
    \end{center}
    \cleardoublepage
}

%%%
%%% Abstract page.
%%%
\newcommand{\thesisabstract}{
    \thispagestyle{empty}
    \begin{center}
    \textbf{\LARGE Preface} \\[1cm]
     \textbf{\LARGE \thesistitle} \\[1cm]
     \textbf{\LARGE \thesisauthor} \\[1cm]
    \end{center}
    \input{preface}
    \cleardoublepage
}

%%%
%%% Miscellaneous
%%%
\newcolumntype{d}[1]{D{.}{.}{#1}}
\newcommand{\draft}{
    \renewcommand{\normalspace}{\singlespace}
    \normalspace
    \chapter*{Draft. Version: \today}
\clearpage }


\definecolor{codegreen}{rgb}{0,0.6,0}
\definecolor{codegray}{rgb}{0.5,0.5,0.5}
\definecolor{codepurple}{rgb}{0.58,0,0.82}
\definecolor{backcolour}{rgb}{0.95,0.95,0.92}

\lstdefinestyle{mystyle}{
    backgroundcolor=\color{backcolour},
    commentstyle=\color{codegreen},
    keywordstyle=\color{magenta},
    numberstyle=\tiny\color{codegray},
    stringstyle=\color{codepurple},
    basicstyle=\ttfamily\footnotesize,
    breakatwhitespace=false,
    breaklines=true,
    captionpos=b,
    keepspaces=true,
    numbers=left,
    numbersep=5pt,
    showspaces=false,
    showstringspaces=false,
    showtabs=false,
    tabsize=2
}

\lstset{style=mystyle}
